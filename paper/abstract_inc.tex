% $Id: abstract.tex 27205 2012-01-23 02:44:21Z karl $

\begin{abstract}

This paper presents experiences using an early development
environment release of the forthcoming Intel MIC platform, focusing on
porting of existing scientific applications and
micro-kernels.  Fortran and C++ applications are chosen
from disciplines including quantum mechanics, hypersonics, rarefied
gas dynamics, finite-element analysis, and FFT and linear algebra
kernels used in the direct numerical simulation of turbulence.
%Multiple MIC programming models are explored across these varied
%applications including automatic offload via the Math Kernel Library
%(MKL), native threaded execution via OpenMP, and shared-memory offload
%mechanisms via newly available compiler directives. 
To characterize the porting efforts required for applications at
different stages of development, codes considered include
both serial applications (no previous threading), codes with high BLAS
usage, and codes with prior shared memory parallelism.


\keywords{co-processors \and heterogeneous computing \and scientific
  applications \and many integrated cores}

\end{abstract}
