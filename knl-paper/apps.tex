\section{Applications and Scientific Kernels}
\label{sec:apps}

The subsections which follow highlight two things: 1) the level of effort
required in porting software applications from a wide array of scientific
disciplines written in commonly used procedural languages; and 2) observed
performance of these applications.  Applications are drawn from four distinct
scientific disciplines:

\paragraph{Incompressible fluid dynamics}  PoongBack is a Fortran application
targeted towards solving the incompressible Navier-Stokes equations in a
periodic wall-bounded channel.  PoongBack uses a Fourier spectral
discretisation in the periodic directions and a B-spline discretisation in the
wall-bounded direction.

\paragraph{Uncertainty quantification}  QUESO is a C++ library for the
Quantifying Uncertainty in Estimation, Simulation, and Optimization.  It
provides a suite of algorithms for sampling unknown probability distributions
and provides a parallel (MPI) environment to allow the user to leverage this in
large-scale engineering applications.  We evaluate a specific likelihood
function called a Gaussian Process surrogate and leverage OpenMP and MKL for
the dense linear algebra involved.

\paragraph{Computational chemistry}  CFOUR is a collection of Fortran
applications that leverages a suite of algorithms for solving coupled-cluster
problems in quantum chemistry.

\paragraph{Finite element methods}  ArcSyn3sis is a C++ library that implements
a blended isogeometric discontinuous Galerkin (BIDG) method that seamlessly
blends the discontinuous Galerkin method with isogeometric analysis.

Given finding from 1) the level of effort required porting this software; and
2) observed performance, our goal will be to comment on the feasibility of the
use of accelerators for doing computational science.  The of this section will
explore each of the application areas mentioned above.
