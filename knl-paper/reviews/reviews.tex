Dear Nicholas,

Congratulations! We are delighted to inform you that your PEARC17 Paper
Submission  Experiences Porting Scientific Applications to the Intel
(KNL) Xeon Phi Platform has been accepted. Attached are the reviews of
your paper that you should address before submitting your final paper
for the proceedings.

You will be notified soon about the schedule for your presentation at
the conference. At least one author must register and attend the
conference for the presentation. Please reserve your hotel room by May
31st to guarantee the conference rate, prices will go up considerably
after this date. The basic schedule is available here:
https://www.pearc.org/schedule

There are a few things that need to be followed up on:

1.       As an accepted paper, you must prepare a final version of your
paper using the latest version of the ACM SIG proceedings templates,
available at
http://www.acm.org/publications/proceedings-template. Please follow the
template instructions precisely. Corresponding authors will also receive
email from the ACM rights management system (rightsreview@acm.org) with
copyright information and other instructions. Upon completion of the
online form, the system will email you the appropriate text for the
permissions block of your final paper.
2.       Presenting Author Information – in order to make sure we are
corresponding with the correct author please let us know which of the
authors will be presenting the paper and their contact
information. Future correspondences regarding the conference attendance
and presentation expectations will be communicated via this author (or
authors). We would like the name and a short bio of the author.
3.       We will feature a fast-forward preview of the paper sessions
after the Tuesday and Wednesday plenaries. This will give you an
opportunity to preview your work to the entire conference. You will have
30 seconds to present one slide summarizing the work, so focus on what
will excite attendees about your paper. We will be sending out a
template closer to the conference date to prepare and submit a single
slide. Please keep this in mind as the conference nears.

The deadline for providing a final paper, with permissions block is
Friday June 9th. This deadline MUST be met in order to get your paper
published.

For additional questions or concerns please email me DIRECTLY at
jwilgenb@umn.edu.

I look forward to seeing you in New Orleans!

Sincerely,
James Wilgenbusch
PEARC17 Accelerating Discovery in Scholarly Research track


----------------------- REVIEW 1 ---------------------
PAPER: 82
TITLE: Experiences Porting Scientific Applications to the Intel (KNL)
Xeon Phi Platform
AUTHORS: Nicholas Malaya, Damon McDougall, Christopher S. Simmons, Craig
Michoski and Myoungkyu Lee

Overall evaluation: 2 (accept)

----------- Overall evaluation -----------
In this paper, the authors detail their efforts porting four scientific
software applications to the Intel Knights Landing (KNL) architecture
that will make up the forthcoming Stampede II supercomputer.  The
applications ported include Poongback (DNS simulations), QUESO
(uncertainty quantification), ArcSyn3sis (a finite element package) and
CFOUR (coupled cluster methods for computational chemistry).  In
general, for each of the packages, they describe the level of effort
required to move from an x86 CPU-based architecture to the KNLs and the
scalability achieved using a variety MPI to OpenMP ratios.

Overall, the authors find that the level of effort required to “port”
these applications was simply a recompile with some additional flags
included to indicate to the compiler the use of a KNL architecture.  The
optimal ratio between MPI ranks and OpenMP threads is generally based on
the algorithmic characteristics of each application.  This is extremely
useful information for other application scientists as it is anticipated
that modern architectures such as the KNL will increasingly be used in
future supercomputers.

The authors could strengthen their paper by providing additional
performance comparisons to existing (CPU) architectures (e.g. Intel
Haswell or Broadwell chips).  While understanding the on-chip
scalability is useful, many application scientists are more interested
in the performance gains that can (or cannot) be achieved with the KNL
chips compared to existing CPU chips.  The comparisons in the CFOUR
application were useful and the authors should be encouraged to expand
this analysis to the other applications.

Minor question: In the CFOUR section, the reference CPU architecture was
two Intel Xeon E5-2670 CPUs (each with 12 cores).  Yet later in the
section, it is mentioned that the OMP_NUM_THREADS was set to 16 (or best
performance was achieved with 16 threads).  Is this per chip or is it 8
per chip?  If the former, this seems surprising if the chip was
oversubscribed (and not in a uniform way).  If the latter, some
explanation of the idle cores may be helpful (e.g. the application was
memory bandwidth limited).  Or it may be that this was a typo because I
would expect the best performance with 12 threads per chip.


----------------------- REVIEW 2 ---------------------
PAPER: 82
TITLE: Experiences Porting Scientific Applications to the Intel (KNL)
Xeon Phi Platform
AUTHORS: Nicholas Malaya, Damon McDougall, Christopher S. Simmons, Craig
Michoski and Myoungkyu Lee

Overall evaluation: 2 (accept)

----------- Overall evaluation -----------
Authors present their effort to port four scientific codes to the KNLs
on the Stampede machine at TACC. The codes are for study of turbulence,
analysis of uncertainty, implementation of FEM methods, and solving
quantum chemistry problems. For all of the codes they thoroughly explain
the optimization efforts and analyze the performance using various
memory hierarchy (that KNL allows), and parallel options (MPI,
OpenMP). As KNLs are new architecture, their analysis of results of
different types of scientific codes would be interest to the community.


----------------------- REVIEW 3 ---------------------
PAPER: 82
TITLE: Experiences Porting Scientific Applications to the Intel (KNL)
Xeon Phi Platform
AUTHORS: Nicholas Malaya, Damon McDougall, Christopher S. Simmons, Craig
Michoski and Myoungkyu Lee

Overall evaluation: 2 (accept)

----------- Overall evaluation -----------
Good paper describing early experiences in porting codes to Intel’s Xeon
Phi MIC platform, Knight's Landing (KNL). They examine a number of
different and important scientific kernels and applications in a variety
of languages and present strong scaling results for all of these and
discuss the effort involved in porting them to KNL. I think there will
be a lot of interest in this topic among attendees. As a personal
quibble, I would like to have seen more results comparing the overall
performance, time to solution, for the given amount of effort, with
codes run on just Intel Xeon chips (i.e. without co-processors) or on
other architectures. I suppose this is effort for future work. :)