\subsection{EOM-CCSDT single-node performance}
\label{sec:cfour}

For the calculations used in this work,
the CFOUR \cite{cfour:08} program was used. CFOUR
(Coupled-Cluster techniques for Computational Chemistry) is an
application for performing high-level quantum chemical calculations
that is under active development by research groups at UT Austin and
Universit\"{a}t Mainz, Germany). CFOUR's major strength is its arsenal
of high-level ab initio methods for the calculation of atomic and
molecular properties.  Virtually all approaches based on M\o
ller-Plesset (MP) perturbation theory and the coupled-cluster
approximation (CC) are available.

For the other applications considered here, the scalibility of various code bases 
used in the PECOS center on KNL compute nodes has been presented. To compliment
this effort, this section will focus on a comparison of optimal single-node performance
between traditional (Haswell) Intel CPUs and the new KNL compute nodes of Stampede 2.
While the scalability of CFOUR on current and emerging architectures will be the focus 
of future work, the results presented here will detail overall runtime performance gains
from migrating to this new architecture. 

For this work, the performance of a single calculation, the EOM-CCSDT energy of the first
excited singlet state of $C_2H_2$, was studied in detail. This excited state is the focus
of a current study being conducted by authors of this paper in which over 1 million single point
energies will be calculated. As such, ensuring the best performance for this run type will conserve
important national compute resources available at TACC. All EOM-CCSDT calculations were performed 
using the NCC module \cite{ncc:15} of the CFOUR program system 
and the ANO1 basis set \cite{ano1:87} with core electrons uncorrelated. A nonequilibrium geometry
of this excited singlet state was used that has C1 symmetry. The lowest symmetry point group available to 
$C_2H_2$ was used because this computational requires the most memory and compute time. This establishes
an upper bound per computation.

Porting CFOUR to the KNL architecture was very straightforward. No modification of the code or the build
system was required. All that needed to be done was to add the previously discussed compile and link flags
to the corresponding Fortran, C, and C++ environmental variables.

The NCC module in CFOUR is a recent contribution to the code and, as such, has been designed to take
advantage of many-core architectures. It uses OpenMP for thread-based parallelism and in this work, the
built-in parallelism of Intel's MKL is also exploited. As has been discussed in detail previously \cite{ncc:15},
it has been determined that on traditional dual Intel CPU compute node systems, best performance is obtained 
by using $OMP\_NUM\_THREADS=16$. This relies on the default behavior that MKL will use $OMP\_NUM\_THREADS$ 
if outside an OpenMP block but run single-threaded within an OpenMP block. In the NCC module, there are MKL-library 
calls within an OpenMP block, however, best performance is achieved by using only serial MKL calls within OpenMP
blocks. This ensures that thread oversubscription never occurs.

For benchmark comparison purposes, the traditional CPU used for reference was an Intel Xeon  E5-2670 v3 CPU
running at 2.30GHz. This CPU was first available on market in late 2014 and is representative of typical HPC 
compute nodes in current generation supercomputers. Best performance on this system was achieved with 
$OMP\_NUM\_THREADS=16$ and the total compute time for the $C_2H_2$ excited state energy discussed previously
was 607 s. If more than 16 threads were used, the total walltime for the calculation increases.

\subsubsection{memory mode and membind}
\subsubsection{cluster mode}
\subsubsection{threads and mkl and environmental variables oh my}
\subsubsection{cfour conclusion}
