\subsection{EOM-CCSDT single-node performance}
\label{sec:cfour}

For the calculations used in this work,
the CFOUR \cite{cfour:08} program was used. CFOUR
(Coupled-Cluster techniques for Computational Chemistry) is an
application for performing high-level quantum chemical calculations
that is under active development by research groups at UT Austin and
Universit\"{a}t Mainz, Germany). CFOUR's major strength is its arsenal
of high-level ab initio methods for the calculation of atomic and
molecular properties.  Virtually all approaches based on M\o
ller-Plesset (MP) perturbation theory and the coupled-cluster
approximation (CC) are available.

For the other applications considered here, the scalibility of various code bases 
used in the PECOS center on KNL compute nodes has been presented. To compliment
this effort, this section will focus on a comparison of optimal single-node performance
between traditional (Haswell) Intel CPUs and the new KNL compute nodes of Stampede 2.
While the scalability of CFOUR on current and emerging architectures will be the focus 
of future work, the results presented here will detail overall runtime performance gains
from migrating to this new architecture. 

For this work, the performance of a single calculation, the EOM-CCSDT energy of the first
excited singlet state of $C_2H_2$, will be studied in detail. This excited state is the focus
of a current study being conducted by authors of this paper in which over 1 million single point
energies will be calculated. As such, ensuring the best performance for this run type will conserve
important national compute resources available at TACC. All EOM-CCSDT calculations were performed 
using the NCC module \cite{ncc:15} of the CFOUR program system 
and the ANO1 basis set \cite{ano1:87} with core electrons uncorrelated.

Porting CFOUR to the KNL architecture was very straightforward. No modification of the code or the build
system was required. All that needed to be done was to add the previously discussed compile and link flags
to the corresponding Fortran, C, and C++ environmental variables.

\subsubsection{previous best practice and haswell performance}
\subsubsection{memory mode and membind}
\subsubsection{cluster mode}
\subsubsection{threads and mkl and environmental variables oh my}
\subsubsection{cfour conclusion}
