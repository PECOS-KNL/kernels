\subsection{EOM-CCSDT single-node performance}
\label{sec:cfour}

For the calculations used in this work,
the CFOUR \cite{cfour:08} program was used. CFOUR
(Coupled-Cluster techniques for Computational Chemistry) is an
application for performing high-level quantum chemical calculations
that is under active development by research groups at UT Austin and
Universit\"{a}t Mainz, Germany. CFOUR's major strength is its arsenal
of high-level {\emph ab initio} methods for the calculation of atomic and
molecular properties.  Virtually all approaches based on M\o
ller-Plesset (MP) perturbation theory and the coupled-cluster
approximation (CC) are available.

This section will focus on a comparison of optimal single-node performance
between traditional (Haswell) Intel CPUs and the new KNL compute nodes of Stampede 2.
While the scalability of CFOUR on current and emerging architectures will be the focus 
of future work, the results presented here will detail overall runtime performance gains
from migrating to this new architecture. 

For this work, the performance of a single calculation, the EOM-CCSDT energy of the first
excited singlet state of $C_2H_2$, was studied.
%\cite{}\todo{can you point to an appropriate paper detailing this state?}. 
%this refers to this work
This excited state is the focus of currently ongoing work being conducted by authors of this paper 
in which over 1 million single point energies on a potential energy surface will be calculated.
%This excited state requires over one million single point
%energies on a potential energy surface to be calculated. 
%clarification: this paper is one single point calculation. other work being done requires 1 million
%of these calculations. This is the motivation for ensuring the best performance of this calculation type
As such, ensuring the best performance for this run type conserves important national compute resources 
available at TACC.  All EOM-CCSDT calculations were performed 
using the NCC module\cite{ncc:15} of the CFOUR program system and the ANO1 basis set\cite{ano1:87} with core 
electrons uncorrelated. The nonequilibrium geometry used for this excited singlet state has C1 symmetry. 
This lowest symmetry point group available to $C_2H_2$ was used because it requires the most memory 
and compute time. This establishes an upper bound per computation.

Porting CFOUR to the KNL architecture was straightforward. No modification of the code or the build
system was required. The compile and link flags previously discussed in Section~\ref{sec:cross_compile}
were added to corresponding Fortran, C, and C++ environmental variables.

The NCC module in CFOUR is a recent contribution to the code and, as such, has been designed to take
advantage of many-core architectures. It uses OpenMP for thread-based parallelism and in this work, the
built-in parallelism of Intel's MKL is also exploited. As discussed previously~\cite{ncc:15},
traditional dual Intel CPU compute node systems found that best performance is obtained 
by using OMP\_NUM\_THREADS of 16. This relies on the default behavior that MKL will use OMP\_NUM\_THREADS 
if outside an OpenMP block but run single-threaded within an OpenMP block. 
In the NCC module, there are MKL-library calls both within and outside OpenMP blocks, and best performance is 
achieved by using only serial MKL calls within OpenMP blocks. This ensures that thread oversubscription never occurs.

For benchmark comparison purposes, the traditional CPUs used for reference were Two Intel Xeon E5-2670 v3 CPUs
running at 2.30GHz. These CPUs were first available on market in late 2014 and is representative of typical HPC 
compute nodes in current generation supercomputers. Best performance on this system was achieved with 
OMP\_NUM\_THREADS of 16 and the total compute time for the $C_2H_2$ excited state energy discussed previously
was 607s.
%\todo{is this hyperthreading? please note hardware cores}. 
% nope 16 hardware threads updated the above text to note it was a dual CPU system
If more than 16 threads were used, the total walltime for the calculation increases. 

While the KNL architecture offers three different memory modes, only two of those are supported on Stampede 2,
cache mode and flat mode. Cache mode preallocates all of the ``fast'' MCDRAM as direct-mapped L3 cache and
as such only 96 GB of DDR4 RAM is presented to the operating system. Flat mode presents both the MCDRAM and the 
DDR RAM to the operating system and the user can use the ``numactl'' utility to decide at run time if memory 
allocations should be directed to the 16 GB of MCDRAM or the 96 GB of DDR RAM. For these comparisons, 16 OpenMP
threads and Intel's recommended default cluster mode of ``Quadrant'' are used.

This mode attempts to localize communication without requiring explicit memory management by the user. 
It does this by grouping tiles into four logical/virtual (not physical) quadrants, then requiring each tile 
to manage MCDRAM addresses only in its own quadrant (and DDR addresses in its own half of the chip). 
This reduces the average number of ``hops'' that tile-to-memory requests require compared to all-to-all mode, 
which can reduce latency and congestion on the mesh. This results in three different runtime configurations:
flat mode with DDR RAM in Quadrant mode (828s), flat mode with MCDRAM in Quadrant mode(789s), and
Cache-Quadrant mode (793s). 

The total memory required for this EOM-CCSDT calculation is 10 GB. As such, the entire calculation can
be fit into the fast MCDRAM. While this does result in a small performance increase (789s vs 828s), it is important
to note that the Cache-Quadrant mode (793s) performed almost identically. This is important for several reasons. First,
the majority of Stampede 2 compute nodes are set up in Cache-Quadrant mode. There are only a handful of Flat nodes
available. Second, many CFOUR calculations will not fit solely in MCDRAM. The Cache-Quadrant mode offers the 
performance benefits of the fast MCDRAM that also applies when a processes uses more than 16 GB. Finally, this
performance benefit is transparent to the enduser. No runtime configuration is required. Because of this, the 
remainder of this section will focus on maximizing performance for Cache-Quadrant systems.

On traditional CPUs, best performance is achieved when using 16 OpenMP threads and not allowing threaded MKL calls
from within an OpenMP block. This is not the case on KNL-based systems. For the EOM-CCSDT energy calculation, 
run time improvements were seen up to 128 OpenMP threads for the NCC module. Additional improvements were seen
when allowing threaded MKL calls from within an OpenMP parallel region. This was done by setting both environmental
variables OMP\_NESTED and MKL\_DYNAMIC to true. The first variable is what enables threaded MKL calls from OpenMP
parallel blocks while the second variable reduces oversubscription from MKL threading within our NCC OpenMP blocks.
This is accomplished by analyzing system workload and dynamically changing the number of MKL threads. This 
combination of environmental variables along with our NCC module results in 128 threads used for NCC OpenMP parallel
regions, 128 MKL threads outside OpenMP blocks and a dynamic number of MKL threads when called within OpenMP blocks.
With these changes, total run time for our EOM-CCSDT calculation was 395s.

Overall, obtaining decent performance with CFOUR on KNL systems was easily achieved. By moving from Haswell to 
KNL, run time performance for the EOM-CCSDT energy of the first excited singlet state of $C_2H_2$ was improved
by 35\%. It is interesting to note, that the parallel scalibility of KNL was better than that of a Haswell-based
system: haswell compute nodes were only able to see performance improvements up to 16 threads, while on KNL,
improvments were seen up to 128 threads. This could be attributed to either the higher memory bandwidth available
on these systems or to improvements introduced with the AVX512 instruction set. Future work will focus on a new
implmentation of the parallelization of NCC to take further advantage of these new systems.

